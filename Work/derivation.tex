\documentclass{report}
\usepackage{algorithmicx, circuitikz, mathtools, float, empheq, xstring, pgfplots, cmupint, enumerate, booktabs, bm, siunitx, physics, enumitem, placeins, environ, xifthen, hyperref, amsmath, textcomp, amssymb, amsfonts, amsthm, geometry, graphicx}
\pgfplotsset{compat=1.18}
\DeclareSIUnit{\mph}{mph}
\DeclareDocumentCommand\vectorbold{ m }{\bm{{#1}}}
\DeclareDocumentCommand\vectorarrow{ m }{\vec{\bm{{#1}}}}
\DeclareDocumentCommand\vectorunit{ m }{\mathop{}\!\IfStrEqCase{#1}{{i}{{\bm{\hat{\textnormal{\bfseries\i}}}}}{j}{{\bm{\hat{\textnormal{\bfseries\j}}}}}}[{\bm{\hat{#1}}}]\mathop{}\!}
\DeclareDocumentCommand\dotproduct{}{\bm{{\cdot}}} % Vector dot product symbol
\DeclareDocumentCommand\crossproduct{}{\bm{{\times}}} % Vector cross product symbol
\DeclareMathOperator{\domain}{dom}
\DeclareMathOperator{\range}{range}
\DeclareMathOperator{\nll}{null}
\DeclareDocumentCommand\dom{}{\trigbraces{\domain}}
\DeclareMathOperator*{\cint}{\cirfnint}
\DeclareDocumentCommand\aqty{ l m }{\braces#1{\langle}{\rangle}{#2}}
\DeclareDocumentCommand\ceil{ l m }{\braces#1{\lceil}{\rceil}{#2}}
\DeclareDocumentCommand\floor{ l m }{\braces#1{\lfloor}{\rfloor}{#2}}
\DeclareDocumentCommand\set{ m o }{\left\{#1\IfNoValueTF{#2}{}{\:\middle|\:#2}\right\}}
\DeclareDocumentCommand\seq{ m o }{\left(#1\IfNoValueTF{#2}{}{\:\middle|\:#2}\right)}
\DeclareDocumentCommand\sstext{ s m }{\IfBooleanTF{#1}{}{\ }\text{#2}\ }
\DeclareDocumentCommand\ss{}{\sstext}
\DeclareDocumentCommand\qst{ s }{\IfBooleanTF{#1}{}{\quad}\text{s.t.}\quad}
\def\idiffd{{\mathchar'26\mkern-11mu \diffd}}
\DeclareDocumentCommand\idd{ o g d() }{\IfNoValueTF{#2}{\IfNoValueTF{#3}{\idiffd\IfNoValueTF{#1}{}{^{#1}}}{\mathinner{\idiffd\IfNoValueTF{#1}{}{^{#1}}\argopen(#3\argclose)}}}{\mathinner{\idiffd\IfNoValueTF{#1}{}{^{#1}}#2} \IfNoValueTF{#3}{}{(#3)}}}
\DeclareMathOperator{\Span}{span}
\newcommand{\suppose}{\makebox[0.5em]{\(_{\rule{.15ex}{.8ex}}\)}\llap{S}}
\newcommand{\N}{\mathbb{N}}
\newcommand{\Z}{\mathbb{Z}}
\newcommand{\Q}{\mathbb{Q}}
\newcommand{\R}{\mathbb{R}}
\newcommand{\I}{\mathbb{I}}
\newcommand{\C}{\mathbb{C}}
\newcommand{\F}{\mathbb{F}}
\newtheorem*{remark*}{Remark}
\newtheorem*{note*}{Note}
\newtheorem*{example*}{Example}
\newtheorem*{exercise*}{Exercise}
\newtheorem*{question*}{Question}
\newtheorem*{theorem*}{Theorem}
\newtheorem*{claim*}{Claim}
\newtheorem*{definition*}{Definition}
\newtheorem*{lemma*}{Lemma}
\newtheorem*{corollary*}{Corollary}
\newtheorem*{recall*}{Recall}
\newtheorem{theorem}[subsection]{Theorem}
\newtheorem{definition}[subsection]{Definition}
\newtheorem{lemma}[subsection]{Lemma}
\newtheorem{corollary}[subsection]{Corollary}
\newtheorem{remark}[subsection]{Remark}
\newenvironment{solution}{\renewcommand{\qedsymbol}{}\proof[Solution]}{\endproof}
\newlist{problems}{enumerate}{3}
\setlist[problems, 1]{label = \textbf{\arabic{problemsi}.}}
\setlist[problems, 2]{label = (\alph{problemsii})}
\setlist[problems, 3]{label = \roman{problemsiii}.}
\def\Course{CRES}
\def\Name{Johan Vonk}
\def\SID{3036855754}
\def\Email{vonk@berkeley.edu}
\def\Title{Derivation}
\def\Session{Fall 2022}
\title{
    \Title\\
    \large \Course\ --\ \Session}
    \author{\Name, \href{mailto:\Email}\Email, \SID}
    \date{\normalsize\today
}
\textheight=9in
\textwidth=6.5in
\topmargin=-.75in
\oddsidemargin=0.25in
\evensidemargin=0.25in
\allowdisplaybreaks{}
\begin{document}
    \maketitle
    \begin{align*}
        \dv{\vb{p}}{t} &= e \pqty{\vb{E} + \frac{\vb{v}}{c} \times \vb{B}} - \frac{2 e^2}{3 c} \vb{g_0}\\
        \vb{g_0} &= \pqty{\frac{e}{mc^2}}^2 \gamma^2 \vb{v} \bqty{\pqty{\vb{E} + \frac{\vb{v}}{c} \times \vb{B}}^2 - \pqty{\frac{\vb{v}}{c} \vdot E}}\\
        E &= \vb{0}\\
        \gamma &= \sqrt{1 + \pqty{\frac{\vb{p}}{m c}}^2}\\
        \vb{v} &= \frac{\vb{p}}{m \cdot \gamma}\\
        &= \frac{\vb{p}}{\sqrt{m^2 + \pqty{\frac{\vb{p}}{c}}^2}}\\
        \implies \vb{g_0} &= \pqty{\frac{e}{mc^2}}^2 \pqty{1 + \pqty{\frac{\vb{p}}{m c}}^2} \frac{\vb{p}}{\sqrt{m^2 + \pqty{\frac{\vb{p}}{c}}^2}} \frac{1}{c} \pqty{\frac{\vb{p}}{\sqrt{m^2 + \pqty{\frac{\vb{p}}{c}}^2}} \times \vb{B}}^2\\
        &= \frac{e^2}{m^2 c^4} \frac{m^2 + \pqty{\frac{\vb{p}}{c}}^2}{m^2} \frac{\vb{p}}{\pqty{m^2 + \pqty{\frac{\vb{p}}{c}}^2}^{\frac{3}{2}}} \frac{1}{c} \pqty{\vb{p} \times \vb{B}}^2\\
        &= \frac{e^2}{m^2 c^5} \frac{\vb{p}}{\sqrt{m^2 + \pqty{\frac{\vb{p}}{c}}^2}} \pqty{\vb{p} \times \vb{B}}^2\\
        &= \frac{e^2}{m^2 c^4} \frac{\vb{p}}{\sqrt{m^2 c^2 + \vb{p}^2}} \pqty{\vb{p} \times \vb{B}}^2\\
        \implies \dv{\vb{p}}{t} &= e \pqty{\frac{\vb{p}}{\sqrt{m^2 c^2 + \vb{p}^2}} \times \vb{B}} - \frac{2 e^2}{3 c} \vb{g_0}\\
        &= \frac{e}{\sqrt{m^2 c^2 + \vb{p}^2}} \pqty{\vb{p} \times \vb{B}} - \frac{2 e^2}{3 c} \vb{g_0}\\
        &= \frac{e}{\sqrt{m^2 c^2 + \vb{p}^2}} \pqty{\vb{p} \times \vb{B}} - \frac{2 e^2}{3 c} \bqty{\frac{e^2}{m^2 c^4} \frac{\vb{p}}{\sqrt{m^2 c^2 + \vb{p}^2}} \pqty{\vb{p} \times \vb{B}}^2}\\
        &= \frac{e}{\sqrt{m^2 c^2 + \vb{p}^2}}  \bqty{\pqty{\vb{p} \times \vb{B}} - \frac{2 e}{3 c} \pqty{\frac{e^2}{m^2 c^4} \vb{p}} \pqty{\vb{p} \times \vb{B}}^2}\\
        \vb{B} &= \pmqty{B & 0 & 0}\\
        \vb{p} &= \pmqty{p_x & p_y & p_z}\\
        \vb{p}^2 &= \sqrt{p_x^2 + p_y^2 + p_z^2}\\
        \vb{p} \times \vb{B} &= \pqty{\vb{p}_{\perp} + \vb{p}_{\parallel}} \times \vb{B}\\
        &= \vb{p}_{\perp} \times \vb{B}\\
        &= \pmqty{0 & p_z B & - p_y B}\\
        \pqty{\vb{p} \times \vb{B}}^2 &= \pqty{\norm{\vb{p}_{\perp}} \norm{B}}^2\\
        &= \pqty{p_y^2 + p_z^2} B^2\\
        \implies \dv{\vb{p}}{t} &= \frac{e}{\sqrt{m^2 c^2 + p_x^2 + p_y^2 + p_z^2}} \bqty{\pmqty{0 & p_z B & - p_y B} - \frac{2 e}{3 c} \pqty{\frac{e^2}{m^2 c^4} \pmqty{p_x & p_y & p_z}} \pqty{p_y^2 + p_z^2} B^2}\\
        &= \frac{e}{\sqrt{m^2 c^2 + p_x^2 + p_y^2 + p_z^2}} \bqty{\pmqty{0 & p_z B & - p_y B} - \pmqty{p_x & p_y & p_z} \frac{2 e^3 \pqty{p_y^2 + p_z^2} B^2}{3 m^2 c^5}}\\
        \dv{\vb{p}}{t} &= \pmqty{\dv{p_x}{t}, \dv{p_y}{t}, \dv{p_z}{t}}\\
        \dv{p_x}{t} &= \frac{e B}{\sqrt{m^2 c^2 + p_x^2 + p_y^2 + p_z^2}} \frac{- 2 e^3 B p_x \pqty{p_y^2 + p_z^2}}{3 m^2 c^5}\\
        \dv{p_y}{t} &= \frac{e B}{\sqrt{m^2 c^2 + p_x^2 + p_y^2 + p_z^2}} \bqty{p_z - \frac{2 e^3 B p_y \pqty{p_y^2 + p_z^2}}{3 m^2 c^5}}\\
        \dv{p_z}{t} &= \frac{e B}{\sqrt{m^2 c^2 + p_x^2 + p_y^2 + p_z^2}} \bqty{- p_y - \frac{2 e^3 B p_z \pqty{p_y^2 + p_z^2}}{3 m^2 c^5}}
        \intertext{Or without the \(g_0\) term,}
        \dv{p_x}{t} &= 0\\
        \dv{p_y}{t} &= \frac{e B}{\sqrt{m^2 c^2 + p_x^2 + p_y^2 + p_z^2}} p_z\\
        \dv{p_z}{t} &= - \frac{e B}{\sqrt{m^2 c^2 + p_x^2 + p_y^2 + p_z^2}} p_y
        \intertext{Solving, we get the solution,}
        p_x(t) &= c_1\\
        p_y(t) &= \pm \sqrt{2c_{2}} \frac{f(t)}{\sqrt{f(t)^{2}+1}}\\
        p_z(t) &= \pm \sqrt{2c_{2}-\frac{2\ c_{2}f(t)^{2}}{f(t)^{2}+1}}
        \intertext{Where,}
        f(t) &= \tan(c_{2}-\frac{eBt}{\sqrt{m^2 c^{2}+c_{1}^{2}+2c_{2}}})
        \intertext{Which is just the equation of a circle.}
    \end{align*}
\end{document}